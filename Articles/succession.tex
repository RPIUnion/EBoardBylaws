\article{Succession}
\begin{enumerate}

    \item If the President is removed from office, has an extended leave of absence, resigns, or is otherwise unable to discharge the powers and duties of office, then the Vice President shall assume the duties of the President as Acting President of the Union.

    \item If the Vice President is unable to discharge the powers and duties of President as Acting President, then the same shall be discharged by the most senior Executive Board Representative able to discharge such duties. Seniority, for the purpose of this Article, shall be defined as the number of calendar months served as an Executive Board Representative. Seniority of Executive Board Representatives with the same number of months served shall be determined by their office in the following order:
    \begin{enumerate}
        \item Policies Committee Chair
        \item Senior Class Representative
        \item Junior Class Representative
        \item Graduate Class Representative
        \item Sophomore Class Representative
        \item Freshman Class Representative
        \item Senate Executive Board Liaison
    \end{enumerate}
    If seniority cannot be determined by office, then seniority shall be determined among Executive Board Representatives with the same number of calendar months served by the order in which they were confirmed by the Student Senate. If no Representative can discharge the powers and duties of President as Acting President, the Grand Marshal shall appoint a Member of the Union as Acting President.

    \item The Acting President of the Union shall discharge the powers and duties of President until such a time that a meeting of the Executive Board can be called. The first and only order of business of such a meeting shall be the selection of a new President of the Union. This meeting shall be presided over by the Judicial Board Chairman, who shall cast no vote in any case. This meeting shall be held no later than two weeks after the President’s office was vacated.

    \item All nominations for the position of President must be moved and seconded by an Executive Board Representative.

    \item Every nominee for the position of President must be a Representative of the Executive Board as defined in these Bylaws.

    \item In the case of multiple nominees, the appointment for President of the Union shall be selected by a series of simple majority votes.
    \begin{enumerate}
        \item If there are two nominees, a vote shall be taken, and the nominee with a simple majority shall be named the appointment for President of the Union.
        \item If there are three or more nominees, a series of votes shall be taken.
        \item In each vote, the nominee with the fewest number of votes shall be removed from consideration, and only the remaining nominees shall be considered in the next vote.
        \item This series of votes shall continue until a single nominee remains, and that nominee shall be named the appointment for President of the Union.
    \end{enumerate}

    \item The nominee who is appointed for President of the Union must be approved by a $\frac{2}{3}$ vote of the total voting membership of the Executive Board. If the appointment is not approved by a $\frac{2}{3}$ vote of the total voting membership of the Board, a new process of nominations begins.

    \item The Executive Board’s appointment for President shall discharge the powers and duties and of President as Acting President until they have been confirmed by the Student Senate. Upon the new President’s confirmation all officers of the Executive Board are relieved of office.

    \item The new President shall nominate officers of the Executive Board.

\end{enumerate}
